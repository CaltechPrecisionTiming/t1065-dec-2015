
\section{Timing Measurements} 
\label{sec:timing} 

We characterize the timing performance of the CdTe sensor by measuring the timestamps
relative to the MCP-PMT device used as a reference timer. An example of the distribution 
of the timestamp measurement for $100$~GeV electrons after $6$~$\mathrm{X}_{0}$ of absorber
material is shown in Figure~\ref{fig:DeltaT}. We extract the time measurement
resolution from this distribution as the width parameter of a gaussian fit.
In Figure~\ref{fig:TimeResolutionVsEnergy} we show the measured time resolution as a function of the
beam energy. At low beam energy, the time resolution improves with increasing size 
of the signal, and is in the noise dominated regime. However, above $100$~GeV, the time 
resolution no longer improves with increasing signal size, pointing towards some
systematic limitation. We study a number of such factors in 
Section~\ref{sec:systematicLimitations} below.

%Fig: example DeltaT plot
\begin{figure}[htbp] 
\centering
\includegraphics[width=0.49\textwidth]{figures/100GeV_deltaT.pdf} 
\caption{Distribution of the timestamp measurement in the CdTe sensor for a $100$~GeV
electron after $6$~$\mathrm{X}_{0}$ of tungsten absorber. } 
\label{fig:DeltaT} 
\end{figure} 


%Fig: Time resolution vs beam energy
\begin{figure}[htbp] 
\centering
\includegraphics[width=0.49\textwidth]{figures/TimeResolutionVsEnergy.pdf} 
\caption{ The measured time resolution of the CdTe sensor is plotted as a function
of the electron beam energy. } 
\label{fig:TimeResolutionVsEnergy} 
\end{figure} 


To further characterize the timing performance of the CdTe signals, we measure the
risetime, defined as the time for the signal to rise from $10\%$ to $90\%$ of the maximum
amplitude, for various electron beam energies. The distribution of risetime for
$100$~GeV electrons are shown on the left of Figure~\ref{fig:riseTime}. The 
measured risetime as a function of the beam energy is shown on the right of 
Figure~\ref{fig:riseTime}. We observe a risetime that is relatively stable
at around $1.35$~ns.


%Fig: riseTime vs energy
\begin{figure}[htbp] 
\centering
\includegraphics[width=0.49\textwidth]{figures/100GeV_risetime.pdf} 
\includegraphics[width=0.49\textwidth]{figures/RisetimeVsEnergy.pdf} 
\caption{ Left: Distribution of risetime of the CdTe signal for $100$~GeV electrons. 
Right: Risetime of the CdTe signal is plotted as a function of the incident beam energy. } 
\label{fig:riseTime} 
\end{figure} 



\subsection{Studies of Systematic Limitations on Time Resolution}
\label{sec:systematicLimitations}

One of the major systematic effects that have been observed in past
timing studies~\cite{Anderson:2015gha,Ronzhin:2015pba,MCPShowerMaxPaper} is the dependence 
of the time measurement on the amplitude of the signal. On the left of 
Figure~\ref{fig:DeltaTVsAmplitude}, we show the dependence of the timestamp measurement 
on the amplitude of the signal, and observe a relatively mild dependance on amplitude.
On the right of Figure~\ref{fig:DeltaTVsAmplitude}, we show the measured time resolution
as a function of the signal amplitude and we observe a characteristic improvement
in the resolution with increasing signal amplitude at low amplitudes. In the region
below about $500$~mV in amplitude, the impact of the signal-to-noise ratio is still 
the dominant factor.

%Fig: DeltaT vs Amplitude
\begin{figure}[htbp] 
\centering
\includegraphics[width=0.49\textwidth]{figures/100GeV_deltaTVsAmp.pdf} 
\includegraphics[width=0.49\textwidth]{figures/TimeResolutionVsAmplitude.pdf} 
\caption{ Left: The distribution of the signal amplitude in the CdTe sensor and 
the time measured in the CdTe sensor relative to the Photek reference detector
is shown in the color scale. The mean value of the time measured in the CdTe sensor 
as a function of the signal amplitude is shown in the black points. Right: The
time resolution is measured as a function of the signal amplitude. } 
\label{fig:DeltaTVsAmplitude} 
\end{figure} 


We also study the dependence of the timestamp measurement as a function of the geometric
position of the incident beam particle as measured by the wire chambers in 
Figure~\ref{fig:DeltaTVsBeamXY}. A relatively large and linear dependence is observed, 
and this geometric position non-uniformity of the time response adds significantly to the 
time resolution (about $35$~ps). Performing a correction for this non-uniformity improves
the time resolution from $44$~ps to $25$~ps for events with $100$~GeV electrons.
The distribution of timestamps after correcting for the geometric position is shown
in Figure~\ref{fig:DeltaTCorr}. The measured time resolution has a mild dependence on 
the beam particle position and is shown in Figure~\ref{fig:TimeResolutionVsBeamXY}.

%Fig: DeltaT vs Beam Location
\begin{figure}[htbp] 
\centering
\includegraphics[width=0.49\textwidth]{figures/DeltaTVsHorizontalPosition.pdf} 
\includegraphics[width=0.49\textwidth]{figures/DeltaTVsVerticalPosition.pdf} 
\caption{ The distribution of the beam particle position measured by the wire chamber
and the time measured in the CdTe sensor relative to the Photek reference detector
is shown in the color scale. The mean value of the time measured in the CdTe sensor as a function
of the beam particle position is shown in the black points.} 
\label{fig:DeltaTVsBeamXY} 
\end{figure} 

\begin{figure}[htbp] 
\centering
\includegraphics[width=0.49\textwidth]{figures/CdTeTimingResolution_100GeV_PositionCorrected.pdf} 
\caption{Distribution of the timestamp measurement corrected for the geometric position
non-uniformity in the CdTe sensor for a $100$~GeV electron after $6$~$\mathrm{X}_{0}$ of tungsten absorber. } 
\label{fig:DeltaTCorr} 
\end{figure} 



%Fig: Time Resolution vs Beam Location
\begin{figure}[htbp] 
\centering
\includegraphics[width=0.49\textwidth]{figures/TimeResolutionVsBeamHorizontalPosition.pdf} 
\includegraphics[width=0.49\textwidth]{figures/TimeResolutionVsBeamVerticalPosition.pdf} 
\caption{ The time resolution is measured as a function of the horizontal (left) and vertical (right)
beam position. }
\label{fig:TimeResolutionVsBeamXY} 
\end{figure} 


%Fig: Time Resolution vs Wire Bond Location  
\begin{figure}[htbp]
\centering 
\includegraphics[width=0.49\textwidth]{figures/CdTeTimingvsR.pdf} 
\caption{ The time resolution is measured as a function of the transverse distance of the incidenting particle to the wire bond of the sensor. The mean time response depencency shown in Fig.~\ref{fig:TimeResolutionVsBeamXY} has been corrected for.}
\label{fig:TimeResolutionVsBeamXY}
\end{figure}




%Extra studies
%MIP peak
%Charge vs beam location ( can we say anything about signal size on shower periphery? )
