\section{Introduction} 

There has been much recent interest in highly granular calorimeters with 
precision timing capability at the level of $20-30$~ps in light of the High-Luminosity
LHC and future higher energy hadron colliders. In order to probe increasingly
rare interactions, future hadron colliders must provide large 
instantaneous luminosity well above $10^{35}$~$\mathrm{cm}^{-2}\mathrm{s}^{-1}$.
With current accelerator and particle detector capabilities, such a high 
instantaneous luminosity will result in very large amounts
of pileup, exceeding several hundreds of simultaneous inelastic collisions per
bunch crossing. Therefore, the crucial ability to identify the origin 
of the particles produced at the different interaction points will be severely 
degraded. Precision timing detectors can be used to recover the ability to 
discriminate between particles produced by different inelastic collisions.
For beam bunch profiles similar to that of the LHC, a detector 
that can measure the time of arrival of a particle
with a precision of $20-30$~ps can effectively reduce the impact of
pileup by a factor of $5$ to $10$. 

Highly granular calorimeters based on silicon sensors as the active material 
have been the focus of recent interest~\cite{Adloff:2009,Butler:2020886}, due to
radiation hardness considerations as well as maturity of the silicon sensor
technology. In this article, we present results of studies of a 
calorimeter prototype using Cadmium-Telluride (CdTe) sensors as the 
active material. CdTe has been studied extensively in the context
of thin film solar cells and has become a mature and wide-spread
technology~\cite{cdtegeneric}. It has also been used as a radiation
detector for nuclear spectroscopy, and is known to have high
quantum efficiency for photons in the x-ray range of the 
spectrum~\cite{cdtesensorsgeneric,cdtesensors1,cdtesensors2,cdtesensors3}.
This feature is of particular interest in the context of its use
in calorimetery because it would be uniquely sensitive to secondary
electromagnetic shower particles in the keV range. Conventional prototypes 
using silicon or scintillator material are not directly sensitive to such high 
energy shower secondaries. Therefore, the first study of electromagnetic
showers using CdTe sensors has the potential to yield new insight
into the behavior of secondary particles produced within an 
electromagnetic shower with energies in the keV range, and has the potential
to yield an improvement on the energy measurement due to
the additional contribution of the higher energy x-ray photons to which previous
calorimeters were not sensitive.

The recent interest on precision timing has resulted in new studies of 
the timing properties of silicon sensors. These studies have found a time resolution 
at $20$~ps level, provided a sufficiently large signal size
in a variety of applications ranging from calorimetery~\cite{SiliconTiming} to 
charged particle detectors~\cite{santacruz}. The signal formation process
in CdTe sensors are very similar to the process in silicon and has 
similar potential to yield precise timestamps.

In this article, we study the signal response of the CdTe sensor to electromagnetic
showers of varying energies and at different shower depths. We also study the timing
performance of the CdTe sensors for electromagnetic showers.

