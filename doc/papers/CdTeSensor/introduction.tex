\section{Introduction} 

There has been much recent interest in enhancing the timing capability of large 
particle physics collider experiments to the level of $20-30$~ps for each final state particle
reconstructed in the detector. In order to probe increasingly rare high energy particle 
interactions, future hadron colliders must provide large 
instantaneous luminosity well above $10^{35}$~$\mathrm{cm}^{-2}\mathrm{s}^{-1}$.
With current accelerator and particle detector capabilities, such a high 
instantaneous luminosity will result in very large amounts
of simultaneous particle collisions, referred to as pileup. 
For the high luminosity upgrade of the Large Hadron Collider (HL-LHC), pileup is expected to 
exceed 200 inelastic proton-proton collisions per bunch crossing, when two ensembles
of up to $10^{12}$ particles collide at the center of the detectors.
In the LHC, the collisions from each bunch crossing are spread out over a length of about $10$~cm 
along the beam axis direction and have an additional time dispersion of about 150 ps.
Under such conditions, the task to associate particles measured in the detectors to a particular 
proton collision within a bunch crossing becomes very challenging. 
Precision timing detectors can be used to recover the ability to 
discriminate between particles produced by different inelastic collisions~\cite{adielba}.
For beam bunch profiles similar to that of the LHC, a detector that can measure the 
time of arrival of a particle with a precision of $20-30$~ps can effectively reduce the impact of
pileup by a factor of $5$ to $10$. 

Highly granular calorimeters based on silicon sensors as the active material 
have been the focus of recent interest~\cite{Adloff:2009,Butler:2020886}, due to
radiation hardness considerations as well as maturity of the silicon sensor
technology. In this article, we present results of studies of a similar
sampling calorimeter prototype using Cadmium-Telluride (CdTe) sensors as the 
active material. CdTe has been studied extensively in the context
of thin film solar cells and has become a mature and wide-spread
technology~\cite{cdtegeneric}. It is also widely used as a radiation
detector for nuclear spectroscopy, and as a sensor for photons in the X-ray range 
due to its high quantum efficiency in this part of the 
spectrum~\cite{cdtesensorsgeneric,cdtesensors1,cdtesensors2,cdtesensors3}.
This feature is of particular interest in the context of its use
in calorimetry because it would enhance the sensitivity to secondary particles in the keV range, 
a significant component of the electromagnetic shower. Conventional prototypes 
using silicon sensors have limited sensitivity to photons in this energy range. 
CdTe sensors are available with thicknesses of several mm which can further enhance the 
efficiency for X-ray photons as well as the ionization signal yield from charged particles.
Therefore, the first study of electromagnetic
showers using CdTe sensors has the potential to yield new insight
into the behavior of secondary particles produced within an 
electromagnetic shower with energies in the keV range, and has the potential
to yield an improvement on the energy measurement due to
the additional contribution of the higher energy X-ray photons to which 
calorimeters based on silicon sensors are less sensitive to.

The recent interest on precision timing has resulted in new studies of 
the timing properties of silicon sensors. 
These studies have found a time resolution 
at $20$~ps level, provided a sufficiently large signal size
in a variety of applications ranging from calorimetery~\cite{SiliconTiming} to 
charged particle detectors~\cite{santacruz}. The signal formation process
in CdTe sensors are very similar to the process in silicon and has 
similar potential to yield precise time-stamps.

In this article, we study the signal response of the CdTe sensor to electromagnetic
showers of varying energies and at different shower depths. We also study the timing
performance of the CdTe sensors for electromagnetic showers.

