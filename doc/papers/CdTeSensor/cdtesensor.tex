%
%
%
\section{Cadmium Telluride Sensor}
\label{sec:siliconpad}
%Fig: Quantum Efficiency vs wavelength
%Photos of sensor, drawing for the circuit

Our measurements were conducted with a CdTe Schottky type diode purchased from Acrorad \cite{acrorad}. It is $1 cm^2$ in size transversely and 1 mm thick.
It was operated at a bias voltage of 700 V, the dark current was between 3 nA and 6 nA depending on the environmental conditions in the test beam experimental yones.     

\begin{figure}[htbp] 
\centering
\includegraphics[width=0.49\textwidth]{figures/CdTeSensor.png} 
\caption{Cadmium-Tellurid sensor used in the setup. The sensor is a Shotky type diode with a transverse size of $1 cm^2$ and a thicknees of 1 mm. It is biased at 700 V.} 
\label{fig:CdTeSensor} 
\end{figure} 


The sensor was placed in a box made of 0.3 mm copper sheets. 
A high bandwidth amplifier from Hamamatsu with an amplification of 36 dB was used amplify the output signal of the CdTe sensor.
An attenuator of 10 dB was used to attenuate the input signal to the amplifier to adjust the CdTe signal to the dynamic range of the amplifier.
The output of the amplifier was fed into a CAEN digitizer operated at 5 GS/s.
