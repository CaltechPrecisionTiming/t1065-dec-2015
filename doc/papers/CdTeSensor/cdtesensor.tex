%
%
%
\section{Cadmium Telluride Sensor}
\label{sec:siliconpad}
%Fig: Quantum Efficiency vs wavelength
%Photos of sensor, drawing for the circuit
The semi-conducting properties of Cadmium-Telluride has been studied since many decades~\cite{cdtegeneric}, 
in particular in the the context of using the material in photovoltaic applications.
Cadmium-telluride sensor are widely used in X-ray detectors~\cite{cdtesensorsgeneric,cdtesensors2,cdtesensors3}. 
They have also been investigated for synchrotron radaition detectors in accelerator technology~\cite{cdtelhc}.   
In our previous 
studies~\cite{Anderson:2015gha,MCPShowerMaxPaper,Ronzhin201552,SiliconTiming,PixelatedMCP,Anderson:2016ygg,Anderson:2015tia} 
we have demonstrated that increasing the primary sensor signal is crucial to achieve good timing resolutions.  
Cadmium-telluride features a significantly larger efficiency for detecting photons in the $10-100$~keV energy range 
compared to silicon sensors. The higher atomic number of Cadmium and Tellurium, averaging to 48.52 for the 
compound bulk material, results in a higher interaction cross section for photons in this energy range. 
Photons with such energies are abundant in electromagnetic showers~\cite{showercomposition}. 
Furthermore, CdTe sensor are available with thicknesses of $1$~mm and more. 
The path-length of the charged shower particles in the sensor material scales accordingly, 
resulting in a larger primary signal.
%
Our measurements were conducted with a CdTe Schottky type diode purchased from Acrorad~\cite{acrorad}. 
It is $1$~$\mathrm{cm}^{2}$ in transverse size and $1$~mm thick.
It was operated at a bias voltage of $700$~V and the dark current was between $3$~nA 
and $6$~nA depending on the environmental conditions in the test beam experimental 
zones.     
%
\begin{figure}[htbp] 
\centering
\includegraphics[width=0.49\textwidth]{figures/CdTeSensor.png} 
\includegraphics[width=0.49\textwidth]{figures/CdTeSensorBox.png} 
\caption{Left: CdTe sensor used in the setup. The sensor is a Schottky type diode with a transverse size 
of $1$~$\mathrm{cm}^{2}$ and a thickness of $1$~mm. It is biased at $700$~V. 
On the front, left corner of the sensor the wire bond connection 
to the metalized top layer of the sensor can be seen. Right: A photograph of the copper-lined alumnium box
enclosing the CdTe sensor. } 
\label{fig:CdTeSensor} 
\end{figure} 
%
The sensor was placed in a box made of $0.3$~mm aluminium sheets sealed with copper tape. 
The electrical circuit shown in Fig.~\ref{fig:cdtecircuit} was used to connect to the sensor to the bias 
voltage with a standard high voltage cable and the readout electronics using a SMA cable with a feed 
through penetrating the aluminium box.

%
\begin{figure}[htbp] 
\centering
\includegraphics[width=0.49\textwidth]{figures/circuit_CdTe.png} 
\caption{Schematic diagram of the circuit used to polarize and read out the 
CdTe sensor. The circuit and the sensor are enclosed in an aluminium box covered by
copper tape.} 
\label{fig:cdtecircuit} 
\end{figure} 
%
