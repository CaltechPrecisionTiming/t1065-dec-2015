
\section{Discussion and Summary}
\label{sec:summary} 


In this article, we describe the first measurement of high energy 
electromagnetic showers using Cadmium Telluride sensors. The signal response
is consistent with expectations and show ....[to be checked more carefully]...compared to silicon.
These initial results are encouraging and motivate future work on 
more detailed comparisons with simulation and more detailed 
measurements of transverse and longitudinal shower profiles.


The energy loss per $\mathrm{g}/\mathrm{cm}^{3}$ mass density 
for a minimum ionizing particle (MIP) is $1.66$~MeV$\mathrm{g}^{-1}\mathrm{cm}^{2}$~\cite{PDG}. 
For CdTe, with a density of $6.2\mathrm{g}/\mathrm{cm}^{3}$, the energy
loss is $1.03$~keV/$\mu$m. The mean energy to produce an electron hole pair
in CdTe is $4.43$~eV~\cite{Sze,Singh}, resulting in a signal size of
233 electron hole pairs per $\mu$m thickness per MIP, or $37$~fC/mm per
MIP. This signal size for the CdTe sensor is a factor of $2.2$ larger per unit thickness
than for silicon sensors. 
%Based on Figure~\ref{figures/ChargeVsEnergyAt6X0.pdf}, we project that at 32~GeV after $6$~$X_{0}$ 
%of absorber, the total charge collected is $2.3$~pC, which amounts to about $62$ MIP-equivalent
%signals. This is larger than the measured value of $54$ for silicon sensors under the same
%beam energy and absorber thickness conditions. The extra signal may be a result of additional
%sensitivity to keV range photons produced in the electromagnetic shower.




We have measured the rise time for signals in the CdTe sensor to be about $1.3$~ns, which
is relatively fast and suggests that there is potential for excellent time precision.
We observe dependencies of the measured time on the geometric position of the
incident beam particle, which is likely due to differences in the charge collection
path. More detailed studies of this aspect are needed and a more optimal design of the 
readout system is possible. Correcting for these dependencies yield time precision results
of $25$~ps for a single layer CdTe sensor of transverse area $1$~cm$\times$~$1$~cm
sampling the electromagnetic shower of a $200$~GeV electron after $6$ radiation lengths. 
These initial results are encouraging and motivate further in-depth studies in the
future.



%% The Appendices part is started with the command \appendix;
%% appendix sections are then done as normal sections
%% \appendix

%% \section{}
%% \label{}
