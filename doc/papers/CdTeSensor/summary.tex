
\section{Discussion and Summary}
\label{sec:summary} 


In this article, we describe the first measurement of high energy 
electromagnetic showers using Cadmium-Telluride sensors. 
These initial results are encouraging and motivate future work on 
more detailed comparisons with simulation and more detailed 
measurements of transverse and longitudinal shower profiles.


%The energy loss per $\mathrm{g}/\mathrm{cm}^{3}$ mass density 
%for a minimum ionizing particle (MIP) is $1.26$~MeV$\mathrm{g}^{-1}\mathrm{cm}^{2}$ 
%for CdTe. For Silicon it is $1.66$~MeV$\mathrm{g}^{-1}\mathrm{cm}^{2}$~\cite{PDG}. 
%For CdTe, with a density of $5.86\mathrm{g}/\mathrm{cm}^{3}$, the energy
%loss is $0.74$~keV/$\mu$m. The mean energy to produce an electron hole pair
%in CdTe is $4.43$~eV~\cite{Sze,Singh}, resulting in a signal size of
%167 electron hole pairs per $\mu$m thickness per MIP, or $27$~fC/mm per
%MIP. This signal size for the CdTe sensor is a factor of $1.55$ larger per unit thickness
%than for silicon sensors. 
%Based on Figure~\ref{figures/ChargeVsEnergyAt6X0.pdf}, we project that at 32~GeV after $6$~$X_{0}$ 
%of absorber, the total charge collected is $2.3$~pC, which amounts to about $85$ MIP-equivalent
%signals. This is larger than the measured value of $54$ for silicon sensors under the same
%beam energy and absorber thickness conditions. The extra signal may be a result of additional
%sensitivity to keV range photons produced in the electromagnetic shower.

We have measured the rise time for signals in the Schottky type CdTe sensor diode to be about $1.35$~ns, which
is relatively fast and suggests that there is potential for precision timing.
We observe dependencies of the measured time on the geometric position of the
beam particle, which may indicate differences in the charge collection path.
More detailed studies of this aspect are needed and a more optimal design of the 
readout system is possible. Correcting for these dependencies yield time resolutions
of $25$~ps for a single layer CdTe sensor of transverse area $1$~cm$~$\times$~$1$~cm
sampling the electromagnetic shower of electrons with energy above $100$~GeV 
after $6$ radiation lengths of tungsten and lead absorber. These initial results are encouraging and motivate 
further in-depth studies in the future.


%% The Appendices part is started with the command \appendix;
%% appendix sections are then done as normal sections
%% \appendix

%% \section{}
%% \label{}
