
\section{Event Reconstruction and Selection }
\label{sec:reco}

All signals are recorded by the CAEN V1742 digitizer with a sampling time of $200$~ps.
The baseline pedestal for each channel is determined using the time samples outside of
the signal window, and is subsequently subtracted from the signal pulses. An example of a recorded
signal waveform in the CdTe sensor for an electromagnetic shower from a 100 GeV electron is shown 
in Figure~\ref{fig:Pulses}. We did not observe an obvious energy dependency of the pulse shapes on 
the particle energy. The pulses feature two components, an initial faster one lasting about 10 ns 
followed by a component extending slightly beyond the 200 ns time window of our readout system. 
The drift velocity of electrons in CdTe is known to be much higher than for the 
holes \cite{scpulses}, which may cause such a pulse shape. The relative size of 
the two signal components are observed to be independent of the energy of the 
incident electron, and therefore its energy may be determined from the fast 
component alone. Using randomly triggered data, we measured the RMS of 
the noise for the channel reading out the CdTe sensor after the amplifier to be about $1.3$~mV. 

%Fig: Pulse shapes for various energies
\begin{figure}[htbp] 
\centering
\includegraphics[width=0.49\textwidth]{figures/CdTe_pulse.pdf}
\includegraphics[width=0.49\textwidth]{figures/CdTe_pulseZ.pdf} 
\caption{Examples of signal pulse in the CdTe sensor for electrons with energies of $100$~GeV. 
The signal pulse shown was recorded by the CAEN V1742 digitizer after a $10$~dB 
attenuator and a $36$~dB fast amplifier. Right: Zoom in of the example pulse. } 
\label{fig:Pulses} 
\end{figure} 

The total charge collected in each channel is obtained by computing the integral of the pulse
waveform over the full $200$~ns range recorded by the digitizer. 
The timestamp for each signal is reconstructed by fitting the pulse waveform with
an appropriate functional form. For signal pulses from the MCP-PMTs, used as reference timers, 
we fit a Gaussian function to a $1.5$~ns window around the peak of the pulse and extract the 
timestamp $t_{0}$ as the mean parameter of the Gaussian function. For signal pulses from the
CdTe sensor, we fit a linear function to time sample points between $10\%$ and $60\%$ of the pulse
maximum and the timestamp $t_{1}$ is assigned as the time at which the fitted linear function
rises to $30\%$ of the pulse maximum. More details of the timestamp reconstruction can be
found in reference~\cite{Anderson:2015gha}.


For the measurements performed at the H2 beamline, 
based on the results shown in Figure~\ref{fig:BeamSensorPosition} we select events
for which the incident beam particle lies within a region of size $3$~mm by $3$~mm 
about the center of the sensor. For the measurements performed at the T9 beamline,
the resolution of the wire chamber measurement was insufficient to make this 
requirement. We also require that the 
signal in the reference MCP-PMT detector has an amplitude larger than $25$~mV. 
For data collected at the H2 beamline, the MCP-PMT detector is located behind
the absorbers and can discriminate between electrons that shower in the absorber
material and pions that do not. We require that the signal amplitude in the 
MCP-PMT detector is larger than $500$~mV to select a pure sample of electrons. For
data collected at the T9 beamline, the electron selection is performed using
the LYSO scintillating crystal placed behind the absorber material and the CdTe sensor,
as shown in Figure~\ref{fig:BeamSchematicDiagram}. The electromagnetic shower particles produce
scintillation light in the LYSO crystal and are read out by an MCP-PMT. We require that
the signal amplitude in the MCP-PMT coupled to the LYSO crystal is larger than $800$~mV
to select a sample of pure electrons. Furthermore, as the precision of the beam particle position 
measured by the wire chambers at the T9 beamline is relatively poor, we also require
large signals in the $1$~cm~$\times$~$1$~cm scintillator trigger counter, with
amplitude above $150$~mV, to constrain the beam to a smaller geometric region. 
